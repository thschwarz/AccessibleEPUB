%% Einleitung.tex
%% $Id: einleitung.tex 28 2007-01-18 16:31:32Z bless $
%%

\chapter{Introduction}
\label{ch:Introduction}
%% ==============================
% CLEARLY SHOW CONTRIBUTIONS AND LINK THEM TO SECTIONS

\section{Background}

\subsection{Motivation}
Over the last several years, accessibility is becoming a word with ever increasing importance. Over the last several decades several nations have passed regulatory acts which guarantee equal treatment between all people and. For example, in 1994 Germany passed the Accessibility Ordinance (Behindertengleichstellungsgesetz) which meant that no one can be disadvantaged due to their disability.

As a result of the regulatory acts in many nations it was required to have documents being able to be read by everyone. Everyone is supposed to have access to documents.

Making accessible documents in paper form is difficult, because each group needs to have its document printed separately. Blind people need to have documents printed in braille, which requires special equipment. Furthermore it is very heavy. Visually impaired people have document requirement which person to person, so it is very difficult to prepare printed documents for them.

Making documents accessible in electronic form is much easier, as the document can be adjusted to each group's requirements easily. The output is also adjustable so if a blind person prefers using a screen reader to a refreshable braille display they are free to do so with little to no extra effort. Visually impaired people can change font, size and colors with only a few clicks.

Currently, the dominant format for electronic documents are PDFs (Portable Document Format, extension .pdf) and word documents(extension .doc, .docx, .odt). While PDF/UA (PDF/Universal Accessibility) has done much in terms of accessibility, there are still some shortcomings. First of all both formats have a predefined page size. While this is useful for printed documents, a computer screen can rarely display all contents of the document to the detriment of visually impaired people.\cite{EPUBzone} The font size is also predetermined and cannot be changed. While zooming can increase the apparent size of the font, the document width may not fit the screen. Furthermore semantic information normally is missing from the documents. For example, a PDF does not necessarily have the document language defined. A PDF might also not have a predefined reading order which means that the header and footer might be output by the screen reader every page. 

Conversely, this means that an electronic document format with no set document size, containing semantic and structure information and a set reading order would be better suited to meet the demands of accessibility. 

\subsection{EPUB}
EPUB stands for electronic publication and is a format primarily used for books in an electronic format (E-book). The EPUB format was created by the International Digital Publishing Forum (IDPF) and the current version is 3.1, which is a minor update to EPUB 3.\cite{EPUBspecs} EPUB uses XML based formats  like XHTML, and thus also uses the accessibility standards and guidelines already established in many nations like the Web Content Accessibility Guidelines (WCAG). \cite{WCAG} This was done as reading systems can have different screen sizes and the EPUB content can therefore be reflowable. Font type and size can also be changed. Visually impaired people could therefore adjust the document to their preferences. The EPUB 3 specification also contains guidelines for accessibility so these features are built in and not an afterthought.\cite{EPUB3bp}

\subsubsection{EPUB Structure}


\subsection{EPUB 3.1}
The EPUB working group has also made some important changes from EPUB 2 to EPUB 3  to make it more accessible. Equations can now be displayed in MathML, there is better navigation and more support for Cascading Style Sheets (CSS). However, not all of these changes are supported yet by many EPUB reading programs and devices.\cite{EPUB30changes} DAISY (Digital Accessible Information System), the audio substitute for print media for the blind, has now been integrated into EPUB 3.\cite{daisyAccessibility}